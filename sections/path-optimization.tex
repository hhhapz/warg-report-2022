\subsection{Path Optimization Model and Process}
\label{sec:path-optimization}

Path Optimization is handled by IMACS on the ground, and then relayed to the
aircraft using one of the command and control links. In each communication
cycle, IMACS is updated on the position of the aircraft, and will always
re-sequence the entire remainder of the waypoints to the UAS, with a special
"starting" waypoint that contains the last known location of the aircraft by IMACS.
This allows air-side path management to be resistant to large transmission delays.

Task one involves safely navigating around a restricted area while flying from
a starting point to a rendezvous point. To handle rerouting, a directed graph
is used to create a path from the current position to the rejoin point while
minimizing distance and avoiding restricted areas.

Task two involves taxiing passengers at waypoints for fares, with the goal of
profit maximization. To accomplish this, a reinforcement learning algorithm is
employed. The algorithm takes into account the number of passengers and their
pickup and drop-off locations, as well as the distance and time required to
reach each waypoint. It then uses the reinforcement learning algorithm approach
toe learn from past iterations and adjust the route in real-time to maximize
profit. The algorithm takes into account factors such as fuel efficiency, time
constraints, and adapting to any other restrictions that may be be present, in
order to make it more efficient than traditional algorithms.
