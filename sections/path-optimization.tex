\subsection{Path Optimization Model and Process}
\label{sec:path-optimization}

Path Optimization is handled by IMACS on the ground, and then relayed to the aircraft
using one of the command and control links. In each communication cycle, IMACS is
updated on the position of the aircraft, and will always re-sequence the entire
remainder of the waypoints to the UAS, with a special "starting" waypoint that
contains the last location of the IMACS knows the aircraft to be in. This allows
air-side path management to always know the closest path to re-join the sequence in case there's significant delay between transmission cycles.

Task one involves safely navigating around a restricted area while flying from 
a starting point to a rendezvous point. To accomplish this task, a graph is created
with the restricted area (with margin) as the vertex, along with the starting
and rendezvous points. The graph is directed such that any straight path that
intersects with the restricted area is not permitted. The edge weight represents the
distance. Dijkstra's algorithm is then used to calculate the shortest path from the 
starting point to the rejoin point.

Task two involves taxiing passengers at waypoints for fares, with the goal of profit
maximization. To accomplish this, a reinforcement learning algorithm is employed. The
algorithm takes into account the number of passengers and their pickup and drop-off
locations, as well as the distance and time required to reach each waypoint. It then
uses the reinforcement learning algorithm approach toe learn from past iterations
and adjust the route in real-time to maximize profit. The algorithm takes into account
factors such as fuel efficiency, time constraints, and adapting to any other
restrictions that may be be present, in order to make it more efficient than
traditional algorithms.

<IMAGE?>