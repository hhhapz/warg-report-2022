\clearpage

\subsection{Passenger Safety and Acceptance}
\label{sec:safety}

All passengers within the UAS are fully enclosed with the carbon fibre
fuselage. They have individual seats as well as a secure four-point harnesses
to keep them safely restrained during flight. All fuselage internals, including
passenger seats, are fastened in place. The primary hatch on the port side of
the aircraft can be easily opened and closed. Stairs are fixed to the inside of
the door for safe boarding and disembarking in case of emergencies.

In case of an electric failure within the aircraft (such as with the battery),
the fuselage of the aircraft is sufficiently protected in order to keep the
passengers safe from risk of fire and smoke.

All electronic components are placed within cases to prevent damage during
maintenance and the unlikely event of rapid acceleration or deceleration.
Connectors during flight are secured with adhesive such that they will not
shift from vibrations while airborne. Externally, a custom arm/disarm PCBA is
leveraged to ensure the ability to disable the board and electronics. The UAS
flight controller features redundant power supplies, permitting swapping of
batteries without letting the aircraft lose power.

All code running on any embedded component of Icarus strictly follows the MISRA
C coding guidelines to ensure safety, portability, security, and reliability
within their respective systems. The software has been rigorously tested on
multiple airframes and platforms for validation and ensuring robustness.
Furthermore, the flight software is designed with concurrency in mind, to
prevent issues such as deadlocks, race conditions and starvation.

In the case of autopilot malfunction, or the system detecting an unknown state,
a fail-safe mechanism transfers manual control back to a monitoring pilot on
the ground. This ensures the pilot can take over full operation of the aircraft
to prevent accidents and complete the flight. The aircraft also can be manually
overridden into manual tele-operation by pilots on the ground at any time if
necessary. These safety measures are designed to ensure the UAS can be operated
safely, and with redundancy in mind under any unforeseen circumstances related
to the safety of the aircraft.
