\subsection{Risk Analysis}
\label{sec:risk-analysis}

\begin{table}[htpb]
	\centering
	\begin{tabular}{p{0.28\linewidth}  p{0.14\linewidth}  p{0.12\linewidth}
		 p{0.36\linewidth} }

		\textbf{Risk} &
		\textbf{Likelihood} & \textbf{Severity} &
		\textbf{Mitigation Technique} \\ \midrule

		Part delivery is much slower, pushing back testing and deadlines. &
		Very likely & Medium & % Likelihod + Severity
		Continue to create and iterate on plans more often, test more parts of
		the aircraft more often and with higher frequency. Book training
		grounds more often, and create more precise deadlines with contingency
		in mind. \\ \midrule

		Pilot failure during competition. &
		Likely & High & % Likelihod + Severity
		Maximize opportunities for pilot training, ensure proper mechanisms for
		communication between pilots and other ground crew, and ensure proper
		briefing before flight. \\ \midrule

		Image recognition failure: The landing pad algorithm fails
		to identify the correct landing pad. & Likely & Low &
		Both ZeroPilot and IMACS have been in development with the possible
		chance of error in mind, and pilots are prepared to take over to
		complete the flight if required. \\ \midrule

		Software failure: The aircraft autopilot fails to operate autonomously
		to complete the flight. &
		Unlikely & High & % Likelihod + Severity
		Manual pilot control will resume for the remainder of the flight, or
		until the autopilot system is operable. In case of failure, manual
		piloting and navigation will be required. \\ \midrule


	    Weather conditions render our aircraft or ground station inoperable. &
		Unlikely & Very high & % Likelihod + Severity
		While designing and developing the aircraft, ensure testing in
		unfavourable conditions for the aircraft, and providing pilots proper
		briefing for flight conditions. \\ \midrule

		Mechanical failure with control surfaces or motors &
		Very Unlikely & High &
		Depending on during which phase of flight this occurs, the
		severity of risk can change. During cruise, attentive pilots can
		switch to multirotor flight and abort the flight with a safe landing.
		During take-off and landing, the window for recovery is extremely low
		and the potential for aircraft damage is likely. Sufficient testing of
		servos and control surfaces will take place before every flight. \\
		\midrule

	\end{tabular}
	\caption{Potential risk that may affect the ability to compete, and
	measures in place to mitigate the effects}
\end{table}

