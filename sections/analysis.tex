\section{Analysis of Mission Requirements}
\label{sec:analysis}

Big City has outlined two principal flight operations for the proposal:
\begin{enumerate}
    \item Task 1: Long Range Passenger Transport
    \item Task 2: On Demand Passenger Transport
\end{enumerate}
Synthesizing the guidelines from both tasks leads to the following requirements:
{
\renewcommand{\arraystretch}{1.1}
\begin{table}[h]
\centering
\begin{tabular}{lrl}
\color{hyperrideblue} \textbf{Requirement} &
\color{hyperrideblue} \textbf{Specification} &
\color{hyperrideblue} \textbf{Reason}                   \\ \midrule
    Maximum weight       & 5 kg           & Required to land on landing pad        \\
	PAX Light            & 360\textdegree{}  visibility, green    & Needed to indicate passengers on-board \\
Max Flight Distance  & > 30 km       & 30 km + re-route                        \\
Max Flight Time      & 60 min        & Maximum time allowed to fly            \\
Top Speed            & 70 km/h       & 30 km in 30-45 mins and takeoff/landing\\  
Max Passengers       & 6 pax         & Task two limitation                    \\
Max Flight Time      & 60 min        & Maximum time allowed to fly with buffer \\
Position Accuracy    & $\pm$ 0.1 m   & Accuracy needed to land on landing pad
\end{tabular}
\caption{Project Icarus core requirements}
\end{table}
}

\subsection{Examination of Solutions}
\label{sec:solutions}

The primary alternate solutions include a fixed-wing, multirotor, and an
airship based aircraft model. Each of these have drawbacks that cause them to
be non-ideal solutions. For instance, fixed-wing systems require a long landing
and takeoff area, which is not ideal given the small-landing pad constraints of
Big City, and can become a safety hazard for civilians living near these
landing pads, as well as the passengers on board the UAS.

In contrast, a multirotor system can easily operate within constrained spaces
for takeoff and landing. However, it cannot perform well in long-distance
travel due to its increased power usage from more motors, and lack of
aerodynamic lift. Based on the requirements set by Big City, there will be
significant distances to cover, reducing the feasibility and efficiency of this
solution.

Finally, an airship would not be usable due to its extremely slow nature and
reliance on lighter-than-air gasses. Moreover, airships have a high number of
failure points, as the likelihood of a gas leak is plausible. Furthermore, its
slow nature reinforces the infeasibility for use as a public transportation
system within a single city.

From the perspective of flight software, ArduPilot is a popular alternative.
Its integrated nature consisting of a flight controller, ground station,
firmware and standardized communication protocol means it is easy to integrate
into a custom airframe. However, use would be limited to predefined protocols
which limits the customizability of the onboard flight software and ground
station.

Hence, considering the benefits and pitfalls of alternate solutions, a hybrid VTOL aircraft which can switch between multirotor mode for takeoff and landing
to fixed-wing operation during long-distance and high speed travel was
selected. Additionally, a custom flight controller with the capability to
support the hybrid flight mode powers the aircraft which gives full control to
implement and configure flight operations and autonomy. Similarly, a fully
in-house ground station complements the flight control software in a way
off-the-shelf components are unable to.
