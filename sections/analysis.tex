\section{Analysis of Mission Requirements}
\label{sec:analysis}

Big City has outlined two principal flight operations for the proposal:
\begin{enumerate}
    \item Task 1: Long Range Passenger Transport
    \item Task 2: On Demand Passenger Transport
\end{enumerate}
Synthesizing the guidelines from both tasks leads to the following requirements:

\begin{table}[h]
\centering
\begin{tabular}{lrl}
\color{hyperrideblue} \textbf{Requirement} &
\color{hyperrideblue} \textbf{Minimum Spec} &
\color{hyperrideblue} \textbf{Reason}                   \\ \midrule
    Maximum weight       & 5 kg           & Required to land on landing pad        \\
PAX Light            & Yes, green    & Needed to indicate passengers on-board \\
Max Flight Distance  & > 30 km       & 30km + re-route                        \\
Max Flight Time      & 60 min        & Maximum time allowed to fly            \\
Position Accuracy    & $\pm$ 0.1 m   & Accuracy needed to land on landing pad
\end{tabular}
\caption{Project Icarus core requirements}
\end{table}

\subsection{Examination of Alternate Solutions}
\label{sec:alternate-solutions}

The primary alternate solutions include a fixed-wing, multirotor, and Zeppelin based
aircraft model. Each of these have significant drawbacks that cause them to be less
than ideal solutions. For instance, fixed-wing systems require a long landing and 
takeoff area, which is not ideal given the small-landing pad constraints of Big City, 
and can become a safety hazard for civilians living near these landing pads, as well
as the passengers on board the UAS.

In contrast, a multirotor system can easily operate within constrained spaces for
takeoff and landing. However, it cannot perform well in long-distance travel due to
its increased power usage from more motors. Based on the requirements set by Big City,
there will be significant distances to cover, reducing the feasibility and efficiency
of this solution.

Finally, a Zeppelin would not be usable due to its extremely slow nature and reliance
on lighter-than-air gasses. Moreover, Zeppelins have a high number of failure points,
as the likelihood of a gas leak is plausible. Furthermore, its slow nature reinforces
the infeasibility for use as a public transportation system within a single city.

From the perspective of flight software, ArduPilot is an alternate solution that has 
been explored. It's integrated nature consisting of a flight controller, ground 
station, firmware and standardized communication protocol (Mavlink), it is easy to
integrate into a custom airframe. However, its use would be limited to predefined 
protocols and limits the customizability of the onboard flight software and ground
station.